\documentclass[a4paper]{article}
\usepackage[top=1in, bottom=1.25in, left=1.25in, right=1.25in]{geometry}
\usepackage{amsmath}
\usepackage{multicol}
\usepackage{graphicx}
\RequirePackage{ltxcmds}[2010/12/07]
%opening
\title{IQ Modulator}

\begin{document}

\maketitle

This blocks takes the two input signals that correspond to the part of the signal in phase and in quadrature and produces a complex signal. In addition it can also produce a binary signal.

It accepts two input signals and it can produce either one or two output signals.

\subsection*{Input Parameters}

\begin{itemize}
	\item outputOpticalPower\{1e-3\} \linebreak
	(double)
	\item outputOpticalWavelength\{1550e-9\} \linebreak (double)
	\item outputOpticalFrequency\{speed$\_$of$\_$light/outputOpticalWavelength\} \linebreak
	(double)
\end{itemize}

\subsection*{Methods}

IqModulator(vector$<$Signal *$>$ \&InputSig, vector$<$Signal *$>$ \&OutputSig) :Block(InputSig, OutputSig)\{\};
\bigbreak
void initialize(void);
\bigbreak
bool runBlock(void);
\bigbreak
void setOutputOpticalPower(double outOpticalPower) \{ outputOpticalPower = outOpticalPower; \}
\bigbreak
void setOutputOpticalPower$\_$dBm(double outOpticalPower$\_$dBm) \{ outputOpticalPower = 1e-3*pow(10, outOpticalPower$\_$dBm / 10); \}
\bigbreak
void setOutputOpticalWavelength(double outOpticalWavelength) \{ outputOpticalWavelength = outOpticalWavelength; outputOpticalFrequency = SPEED$\_$OF$\_$LIGHT / outOpticalWavelength; \}
\bigbreak
void setOutputOpticalFrequency(double outOpticalFrequency) \{ outputOpticalFrequency = outOpticalFrequency; outputOpticalWavelength = outOpticalFrequency / outputOpticalFrequency; \}

\subsection*{Functional Description}

The complex signal is multiplied by $\frac{1}{2}\sqrt{\textit{outputOpticalPower}}$ in order to reintroduce the information about the energy (or power) of the signal. This signal corresponds to an optical signal and it can be a scalar or have two polarizations along perpendicular axis. It is the signal that is transmited to the receptor. 

The binary signal is sent to the Bit Error Rate (BER) meaurement block.

\subsection*{Input Signals}

\textbf{Number}: 2

\textbf{Type}: Sequence of impulses modulated by the filter (ContinuousTimeContiousAmplitude))

\subsection*{Output Signals}

\textbf{Number}: 1 or 2

\textbf{Type}: Complex signal (optical) (ContinuousTimeContinuousAmplitude) or binary signal (DiscreteTimeDiscreteAmplitude)

\subsection*{Example}
%\begin{figure}[h]
%	\includegraphics[width=\textwidth]{MQAM8}
%\end{figure}

\subsection*{Sugestions for future improvement}

\end{document}