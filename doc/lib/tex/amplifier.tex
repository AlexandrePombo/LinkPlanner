\documentclass[../../sdf/tex/BPSK_system.tex]{subfiles}
%opening
\onlyinsubfile{\title{Amplifier}}

\begin{document}

\onlyinsubfile{\maketitle}

\subsection*{Input Parameters}

\begin{multicols}{2}
	\begin{itemize}
		\item setAmplification
		\item setNoiseAmplitude
	\end{itemize}
\end{multicols}

\subsection*{Functional Description}

This block accepts one real signal and outputs one real signal built from multiplying the input signals by a predetermined value. This block also adds random gaussian distributed noise with a user defined amplitude. The multiplying factor and noise amplitude are defined by the values of \textit{Amplification} and \textit{NoiseAmplitude} respectively.

\subsection*{Input Signals}

\textbf{Number}: 1

\textbf{Type}: Real signal (ContinuousTimeContiousAmplitude)

\subsection*{Output Signals}

\textbf{Number}: 1

\textbf{Type}: Real signal (ContinuousTimeContiousAmplitude)

\end{document}