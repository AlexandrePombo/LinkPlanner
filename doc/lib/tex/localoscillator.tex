\documentclass[a4paper]{article}
\usepackage[top=1in, bottom=1.25in, left=1.25in, right=1.25in]{geometry}
\usepackage{amsmath}
\usepackage{multicol}
\usepackage{graphicx}
\usepackage[utf8]{inputenc}
\usepackage[english]{babel}
\setlength{\parskip}{0.03cm plus4mm minus3mm}
\RequirePackage{ltxcmds}[2010/12/07]
%opening
\title{Local Oscillator}

\begin{document}

\maketitle

\subsection*{Input Parameters}

\begin{multicols}{2}
	\begin{itemize}
		\item LocalOscillatorPhase
		\item LocalOscillatorOpticalPower
		\item \verb|LocalOscillatorOpticalPower_dBm|
	\end{itemize}
\end{multicols}

\subsection*{Functional Description}

This blocks accepts a complex signal (either with XY polarization or a simple Band Pass signal) and outputs a phase constant complex signal with the same length as the input signal. The phase and optical power are defined by the values of \textit{LocalOscillatorPhase} and \textit{LocalOscillatorOpticalPower} respectively.

\subsection*{Input Signals}

\textbf{Number}: 1

\textbf{Type}: Complex signal (ContinuousTimeContiousAmplitude)

\subsection*{Output Signals}

\textbf{Number}: 2

\textbf{Type}: Complex signal (ContinuousTimeContiousAmplitude)

\end{document}