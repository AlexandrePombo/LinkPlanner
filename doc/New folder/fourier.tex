\documentclass{article}
\usepackage[utf8x]{inputenc}
\usepackage{amsmath}
\usepackage{float}
\usepackage{graphicx}
\usepackage{multicol}
\usepackage{multirow}
\usepackage[margin=1in]{geometry}
\usepackage{indentfirst}
\usepackage{amsfonts}

\begin{document}

\title{Fourier transform}
\maketitle

There are various possible definitions for the Fourier transform. We shall use the definition presented in the reference~\cite{agrawal1992fiber}, presented in~\eqref{eq:Fourier}.
\begin{subequations}\label{eq:Fourier}
\begin{align}
S(\omega)&=\int_{-\infty}^{\infty}dt~s(t)e^{i\omega t}\\
s(t)&=\frac{1}{2\pi}\int_{-\infty}^{\infty}d\omega~ S(\omega)e^{-i\omega t}.
\end{align}
\end{subequations}

We can also define the n-dimensional definition, presented in~\eqref{eq:ndimensional}.

\begin{subequations}\label{eq:ndimensional}
\begin{align}
S(\omega)&=\int_{\mathbb{R}^n}dt~s(t)e^{i\omega t}\\
s(t)&=\left(\frac{1}{2\pi}\right)^n\int_{\mathbb{R}^n}d\omega~S(\omega)e^{-i\omega t}
\end{align}
\end{subequations}

It may also prove useful to define the the equivalent discrete Fourier transform~\eqref{eq:discrete}, for a N point function.
\begin{subequations}\label{eq:discrete}
\begin{align}
S(\omega_m)=&\sum_{n=0}^{N-1} s(t_n)e^{i\omega_m t_n}\\
s(t_n)=\frac{1}{N}&\sum_{m=0}^{N-1}S(\omega_m)e^{-i\omega_m t_n}
\end{align}
\end{subequations}

\bibliographystyle{unsrt}
\bibliography{bibliography}

\end{document}