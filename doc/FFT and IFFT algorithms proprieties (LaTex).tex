\documentclass[12pt,a4paper,openany]{article}
\usepackage{amsmath}
\usepackage{amsfonts}
\usepackage{amssymb}
\usepackage{graphicx}
\usepackage{cite}
\usepackage{subcaption}
\usepackage{makeidx}
\usepackage[left=2.00cm, right=2.00cm]{geometry} %margens
\usepackage{float}
\usepackage{relsize}
\usepackage[bookmarks,hidelinks]{hyperref}
%\usepackage
% Title Page
\title{Different algorithms of FFT and IFFT}
\author{Uladzislau Zubets}


\begin{document}
\maketitle
\thispagestyle{empty}
\pagebreak
%\begin{abstract}
%\end{abstract}
\tableofcontents
\listoffigures
\thispagestyle{empty}
\pagebreak
\setcounter{page}{1}

\section{FFT (Fast Fourier Transform):}

\hspace{1cm}
1º algorithm (associated to 1o IFFT algorithm):


1.	This algorithm is recursive


2.	Higher memory requirements and redundancy


3.	More intuitive than 2º algorithm


4.	Simply to use and implement


5.	Uses even and odd variables


6.	It has complex Fourier transformation


7.	Requires high number discretization intervals (if frequency of discretization (fd) = frequency of signal (fs), then information is lost; if (fd = 2fs) > fs, then information is save)


8.	It needs O(N$^{2}$) algebraic operations and it has also low propagation error with O(N log N)


9.	It is made by Cooley–Tukey FFT algorithm (in-place, divide-and-conquer: divide even and odd, conquer (calculates individually each one) fft_even and fft_odd)


2º algorithm (associated to 1º IFFT algorithm):


1.	This algorithm is recursive


2.	Better optimized (low memory requirements) than 1o algorithm


3.	Less intuitive than 1o algorithm


4.	It has complex Fourier transformation


5.	More complex (difficulty) than 1º algorithm


6.	Uses even and odd variables


7.	It has complex Fourier transformation


8.	Requires low interval of discretization


9.	It is made by Cooley–Tukey FFT algorithm (in-place, breadth-first, decimation-in-frequency)


3º algorithm (included FFT and IFFT):


1.	This algorithm is recursive


2.	It has complex Fourier transformation


3.	This algorithm has high complexity


4º algorithm:


1.	This algorithm is recursive


2.	It is recommended to be applied to Gauss beam (variable frequency along time axis)


3.	It is necessary extraordinary to be accurate when are computing frequencies


4.	For higher precision is necessary to adjust parameters, to use «correct» window, to use non - recursive FFT


5.	Very fast


6.	This algorithm is adjusted to «wavelets» propriety like as FFTW (variable frequency along time axis)


5º algorithm:


1.	This algorithm is recursive


2.	It is made by Cooley-Tukey algorithm


3.	This algorithm has high complexity


4.	It needs O(N$^{2}$) algebraic operations and it has also low propagation error with O(N log N)


5.	It is recommended to be applied to Gauss beam (frequency impulse changing along time axis)


6.	It has maximum flexibility


7.	Very fast velocity


8.	A complex number class meant to simplify the handling of complex numbers, at some expense of execution speed (Complex.cpp and Complex.h, although can be compiled without Complex.x)


9.	A Discrete Fourier Transform routine, included for its simplicity and educational value. Very slow. Users can invoke this conversion with " ./fft_processor -d" (DFT.cpp and DFT.h, although can be compiled without DFT.x)


10.	A well-optimized Fast Fourier Transform using the Danielson-Lanzcos lemma. This routine requires that the array size be a power of 2. Notice that an array of bit-reversed pointers is created in advance of computation, an optimization that pays off during multiple conversions using the same array size (for a single conversion it has no effect).


11.	Readers familiar with FFT code examples may be surprised by the relative brevity of the main conversion routine. This results from use of a Complex number class (see above) that handles some operations in a way that allows a simplification of the source, probably at the expense of execution speed.


6º algorithm (included FFT and IFFT):


1.	This algorithm is recursive


2.	It is made by Cooley-Tukey algorithm


3.	It has complex Fourier transformation


4.	This algorithm has high complexity


5.	It needs O(N$^{2}$) algebraic operations and it has also low propagation error with O(N log N)


6.	Exist two functions for forward transform and two ones for inverse transform


7.	Every couple consists of in-place version and version that preserves input data and outputs transform result into provided array


8.	Protected section of the class has as well four functions: two functions for data preprocessing — putting them into convenient order, core function for transform performing and auxiliary function for scaling the result of inverse transform


9.	Inside wrapper you can find check of input parameters, then data preparation — rearrangement, — and transform itself


10.	Rearrange function uses our «mirrored mathematics» to define new position for every element and swaps elements


11.	While loop implements addition of 1 in mirrored manner to get target position for the element


12.	Here in-place forward Fourier transform performed for signal of 1024 samples size


13.	There is also Sand-Tukey algorithm that rearranges data after performing butterflies and in its case butterflies look like recursive, but mirrored to the right so that big butterflies come first and small ones do last


14.	From all considerations follows that length of input data for our algorithm should be power of 2. In the case length of the input data is not power of 2 it is good idea to extend the data size to the nearest power of 2 and pad additional space with zeroes or input signal itself in a periodic manner — just copy necessary part of the signal from its beginning. Padding with zeroes usually works well


7º algorithm (included FFT and IFFT):


1.	This algorithm is recursive


2.	It is made by Cooley-Tukey algorithm


3.	It needs O(N$^{2}$) algebraic operations and it has also low propagation error with O(N log N)


4.	It has complex Fourier transformation


5.	This algorithm has high complexity


6.	Makes the «butterfly» transform


7.	Moves elements of the array as required by the iterative FFT implementation


8.	Takes time O(N * log(N)) where N must be a power of 2


%%%%%%%%%%%%%%%%%%%%%%%%%%%%%%%%%%%%%%%%%%%%%%%%%%%%%%%%%%%%%%%%
%%%%%%%%%%%%%%%%%%%%%%%%%%%%%%%%%%%%%%%%%%%%%%%%%%%%%%%%%%%%%%%%
%%%%%%%%%%%%%%%%%%%%%%%%%%%%%%%%%%%%%%%%%%%%%%%%%%%%%%%%%%%%%%%%
%%%%%%%%%%%%%%%%%%%%%%%%%%%%%%%%%%%%%%%%%%%%%%%%%%%%%%%%%%%%%%%%
%%%%%%%%%%%%%%%%%%%%%%%%%%%%%%%%%%%%%%%%%%%%%%%%%%%%%%%%%%%%%%%%
%%%%%%%%%%%%%%%%%%%%%%%%%%%%%%%%%%%%%%%%%%%%%%%%%%%%%%%%%%%%%%%%
%%%%%%%%%%%%%%%%%%%%%%%%%%%%%%%%%%%%%%%%%%%%%%%%%%%%%%%%%%%%%%%%
%%%%%%%%%%%%%%%%%%%%%%%%%%%%%%%%%%%%%%%%%%%%%%%%%%%%%%%%%%%%%%%%
%%%%%%%%%%%%%%%%%%%%%%%%%%%%%%%%%%%%%%%%%%%%%%%%%%%%%%%%%%%%%%%%
%%%%%%%%%%%%%%%%%%%%%%%%%%%%%%%%%%%%%%%%%%%%%%%%%%%%%%%%%%%%%%%%
%%%%%%%%%%%%%%%%%%%%%%%%%%%%%%%%%%%%%%%%%%%%%%%%%%%%%%%%%%%%%%%%
%%%%%%%%%%%%%%%%%%%%%%%%%%%%%%%%%%%%%%%%%%%%%%%%%%%%%%%%%%%%%%%%
%%%%%%%%%%%%%%%%%%%%%%%%%%%%%%%%%%%%%%%%%%%%%%%%%%%%%%%%%%%%%%%%
%%%%%%%%%%%%%%%%%%%%%%%%%%%%%%%%%%%%%%%%%%%%%%%%%%%%%%%%%%%%%%%%
%%%%%%%%%%%%%%%%%%%%%%%%%%%%%%%%%%%%%%%%%%%%%%%%%%%%%%%%%%%%%%%%
%%%%%%%%%%%%%%%%%%%%%%%%%%%%%%%%%%%%%%%%%%%%%%%%%%%%%%%%%%%%%%%%
%%%%%%%%%%%%%%%%%%%%%%%%%%%%%%%%%%%%%%%%%%%%%%%%%%%%%%%%%%%%%%%%
%%%%%%%%%%%%%%%%%%%%%%%%%%%%%%%%%%%%%%%%%%%%%%%%%%%%%%%%%%%%%%%%
%%%%%%%%%%%%%%%%%%%%%%%%%%%%%%%%%%%%%%%%%%%%%%%%%%%%%%%%%%%%%%%%
%%%%%%%%%%%%%%%%%%%%%%%%%%%%%%%%%%%%%%%%%%%%%%%%%%%%%%%%%%%%%%%%
%%%%%%%%%%%%%%%%%%%%%%%%%%%%%%%%%%%%%%%%%%%%%%%%%%%%%%%%%%%%%%%%
%%%%%%%%%%%%%%%%%%%%%%%%%%%%%%%%%%%%%%%%%%%%%%%%%%%%%%%%%%%%%%%%
%%%%%%%%%%%%%%%%%%%%%%%%%%%%%%%%%%%%%%%%%%%%%%%%%%%%%%%%%%%%%%%%
%%%%%%%%%%%%%%%%%%%%%%%%%%%%%%%%%%%%%%%%%%%%%%%%%%%%%%%%%%%%%%%%
%%%%%%%%%%%%%%%%%%%%%%%%%%%%%%%%%%%%%%%%%%%%%%%%%%%%%%%%%%%%%%%%
%%%%%%%%%%%%%%%%%%%%%%%%%%%%%%%%%%%%%%%%%%%%%%%%%%%%%%%%%%%%%%%%
%%%%%%%%%%%%%%%%%%%%%%%%%%%%%%%%%%%%%%%%%%%%%%%%%%%%%%%%%%%%%%%%
%%%%%%%%%%%%%%%%%%%%%%%%%%%%%%%%%%%%%%%%%%%%%%%%%%%%%%%%%%%%%%%%
%%%%%%%%%%%%%%%%%%%%%%%%%%%%%%%%%%%%%%%%%%%%%%%%%%%%%%%%%%%%%%%%
\section{IFFT (Inverse Fast Fourier Transform):}

\hspace{1cm}
1º algorithm (associated to 1º and 2º FFT algorithm):


1.	This algorithm is recursive


2.	Higher memory requirements and redundancy


3.	Simply to use and implement


4.	Uses even and odd variables


5.	It has complex Fourier transformation


6.	Requires high number discretization intervals (if frequency of discretization (fd) = frequency of signal (fs), then information is lost; if (fd = 2fs) > fs, then information is save)


7.	It needs O(N$^{2}$) algebraic operations and it has also low propagation error with O(N log N)


8.	It is made by Cooley–Tukey FFT algorithm (in-place, divide-and-conquer)


\end{document}
