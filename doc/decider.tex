\section{Bit Decider}

\subsection*{Functional Description}

This block accepts one real discrete signal and outputs a binary string, outputting a 1 if the input sample is above the predetermined reference level and 0 if it is below another reference value The reference values are defined by the values of \textit{PosReferenceValue} and \textit{NegReferenceValue}.


\subsection*{Parameters}

\begin{multicols}{2}
	\begin{itemize}
		\item setPosReferenceValue
		\item setNegReferenceValue
	\end{itemize}
\end{multicols}

\subsection*{Input Signals}

\textbf{Number}: 1

\textbf{Type}: Real signal (DiscreteTimeContiousAmplitude)

\subsection*{Output Signals}

\textbf{Number}: 1

\textbf{Type}: Binary (DiscreteTimeDiscreteAmplitude)